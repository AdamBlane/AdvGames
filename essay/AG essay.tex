\documentclass{article}
\usepackage{graphicx}
\usepackage[utf8]{inputenc}
\usepackage[english]{babel}

\usepackage{hyperref}
\hypersetup{
	colorlinks=true,
	linkcolor=blue,
	filecolor=magenta,      
	urlcolor=cyan,
}

\begin{document}

\title{Engine Investigation report}
\author{Adam Blance}
\maketitle

\section{Introduction}
For years many games companies have been using pre-existing engines to help bring their games to life. These engines are becoming more essential as the market increases its reliance on them, therefore it is important to look at these engines more in depth and to consider their uses and impact on the entire games industry.  In this essay some of most popular engines are compared and contrasted. 

\section{Unity}
Currently the most used game engine is Unity it dominates the market with 45\% of all engines being used being Unity. It is being used in games such as: Rust, Hearthstone: Heroes of Warcraft and Kerbal Space Program.
\newline
Unity is all purpose game engine that uses C\# scripting, Although it supports two other programming languages, Unity is also known for its great cross platform support, It currently has support for over twenty five platforms, including but not limited to: Playstation 4 , Nintendo Switch , Xbox One and Oculus Rift. Unity is well known for its comfortable UI with a drag and drop system that allows beginner programmers to use Unity with relative ease. 

\section{Unreal Engine}
The Unreal Engine was first developed back in 1998, since then more iterations of the Unreal engine have come out, the most recent being Unreal Engine 4. In that time many games have used the Unreal Engine such as: Bioshock Infinite,  Borderlands 2, Batman Arkham City and Mass Effect 2. It currently holds the Guinness World record for being the "most successful video game engine."
\newline Created by Epic Games, the Unreal Engine uses the coding language C++ and is completely open source allowing Epic Games to work with their community to be constantly changing and improving their game engine. This is also backed by the large amount of documentation that can be found online.

\section{CryEngine}
Cry engine currently takes up 5\% of the total game engine market. First released back in 2004, it has gone through many changes with the most recent CryEngine being the CryEngine V relased on March 22 2016 by CryTek. Since then many successful games have been released using this engine such as: The Far Cry series, The Crysis Franchise and the Xbox One Exclusive Ryse: Son of Rome.
\newline CryEngine’s main language it uses is C++ although it uses Lua for certain parts of game logic and XML in places.  The CryEngine has some amazing graphical capabilities with advanced animation and state of the art lighting while also being surprisingly cheap. One of the main positives of the CryEngine is its price tag, Crytek offers the engine without royalty commitments. This can be very appealing to beginner developers.  

\section{Unity vs Unreal Engine vs CryEngine}
To compare these three game engines several aspects will be compared. The first of these being ease of use. This is a very important feature of game engines because this takes into account not only how easy it is to use but also how robust the engine is. The obvious answer for which of these is the easiest to use is Unity. Unity is well known for being a comfortable environment to programme in. It also has a great UI that can be quickly picked up. Unreal Engine is also very useful for beginners, with a large amount of documentation online with a large fan base even if you do get stuck on a programming problem it has probably already been solved online. While Unreal Engine might not have the same quality UI as Unity it boasts a larger online community. The final engine to be looked at is the CryEngine. Unfortunately for CryEngine while it might not be difficult to use as the other engines. It lacks the comfortable UI of Unity and the only online documentation that could be found on the CryEngine is on the official website. This is probably because of CryEngines V relative newness compared to the other engines. This means CryEngine cannot compete with Unity or Unreal Engines online community. Between the other two Unity is probably the easiest to use but this is hardly a surprise as this is one of Unitys main selling points.
\newline
The next aspect that will be compared is graphical constraints.

\section{Conclusion}
Of these three game engines I would probably say my preferred game engine to use would be Unreal Engine. Even though I have never used it before and I do have an extensive background with Unity. During research for this essay everything I read about Unreal Engine was positive and I hope to use it in the future. As for the other two both are very useful in their own right and at the end of the day it probably comes down to personal preference. 


\end{document}